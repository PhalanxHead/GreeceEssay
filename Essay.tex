\documentclass[a4paper, 12pt]{article}

%Package Settings
\usepackage[margin=2.5cm, includefoot]{geometry}
\usepackage{setspace}
\doublespacing
\usepackage[backend=biber,authordate]{biblatex-chicago}
\addbibresource{Refs.bib}
\usepackage{graphicx}
\usepackage{csquotes}
\usepackage{fancyhdr}
\usepackage{titling}

% Page Style Stuff
\fancypagestyle{plain}{
  \fancyhf{}
  \renewcommand{\headrulewidth}{0pt}
  \renewcommand{\footrulewidth}{0.4pt}
  \lfoot{\theauthor}
  \cfoot{\thetitle}
  \rfoot{Page \thepage}
}

% Title Formatting
\title{Greek Battle Strategy \\ And Its Victory Over Persia}

\author{Luke Hedt}
\date{\today}

% Handy Citation Notes for later:
% \footcite[, 200]{morris_powell_2010} %To add page numbers to citations
% \blockquote{} % If you plan to quote a huge section of Herodotus

\begin{document}

\begin{titlepage}
    \centering
    \includegraphics[width=0.25\textwidth]{UniLogo.png}\par\vspace{1cm}
    {\scshape\Large THE UNIVERSITY OF MELBOURNE \\
              \large ANCW20022 Ancient Greece: \\
              History and Archaeology Essay\par}
    \vspace{1.5cm}
    {\Huge \thetitle \par}
    \vfill

% Bottom of the page
    {\Large\itshape \theauthor \par}
    \vspace{1.5cm}
    {\Large \today\par}
\end{titlepage}
% Format first page correctly.
\pagestyle{plain}

The early 5th century BC was a pivotal time in the formation of the Hellenic
identity, marking the decisive point between the period of \emph{p{\'o}leis} Greece
and the true Hellenic period. The cornerstone of this identity formation
was the defeat of Persia in the Greco-Persian Wars, beginning in 499 BC
with the Ionian Revolt. But what brought about the defeat of Persia? What tactics
were the Greeks employing in the period that brought about their victories?
This essay will attempt to address these questions, discussing how their
strategies and arms and armour were instrumental in the defeat of Persia
in the Greco-Persian Wars and
Carthage in the First Sicilian War.

\vspace{1em} \par

The Greek City States were not a cohesive unit at the outbreak of the 5th century
BC as we tend to think of them today. Rather each urban centre tended to control
the land around it, and considered itself a separate (and usually superior)
entity to the other cities that would now be considered Greek. Your city of
origin was usually considered more important than their `Greekness,' it was
a more relevant identifier, to be say, Athenian than Greek prior to the
Greco-Persian Wars. However, most of the Greek city states often equipped
themselves rather similarly, copying the armour style of Sparta, the most
effective fighting force in the locality.


\nocite{*}
% Bib page
\newpage
\printbibliography

\end{document}
